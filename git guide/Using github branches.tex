\documentclass[a4paper, titlepage]{article}
\usepackage[utf8]{inputenc}
\usepackage{hyperref}
\usepackage{xcolor}
\hypersetup{
	colorlinks,
	linkcolor={red!50!black},
	urlcolor={blue!80!black}
}
\usepackage{graphicx}
\graphicspath{ {images/} }
\title{Tom's Git Guide}
\author{Tom Schwarz, Ben Schwarz}
\date{Janurary 2019}

\begin{document}
\maketitle
\renewcommand{\abstractname}{Introduction} %changes the abstract to be called an introduction.
\begin{abstract}
This guide was written to explain how to use git (and GitHub) quickly and easily. While it definitely is not a particularly good guide, I couldn't find one online that explained, practically, how to use git with the right context and simply enough. So I wrote this. For an explanation of how Git works (which you \emph{must} know before you read this (knowing ``git commit'', ``git pull'' and ``git push'' or their equivalent GUI buttons dosen't count)) see \href{https://blog.red-badger.com/2016/11/29/gitgithub-in-plain-english}{Git \& Github in plain english}.
\end{abstract}

\tableofcontents

\section{Branching \& Pull Request Workflow}
\subsection{Creating a branch}
temp

\subsection{Creating the pull request}
temp

\subsection{Reviewing the pull request}
temp

\subsubsection{Making changes}
temp

\subsubsection{Approving the pull request}
temp

\subsection{cleanup}
temp

%a test picture, kept here so I can remember how to do it.
\begin{figure}[h]
	\centering
	\includegraphics{test-image}
	\caption{a test image}
\end{figure}

\end{document}
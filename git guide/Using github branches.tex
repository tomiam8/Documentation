\documentclass[a4paper, titlepage]{article}
\usepackage[utf8]{inputenc}
\usepackage{graphicx}
\graphicspath{ {images/} }
\usepackage{xcolor}
\usepackage{hyperref}
\hypersetup{
	colorlinks,
	linkcolor={red!50!black},
	urlcolor={blue!80!black}
}
\title{Tom's Git Guide}
\author{Tom Schwarz, Ben Schwarz}
\date{Janurary 2019\\v1.A} %letters for pre-releases, .0 for main release, .1+ for post-updates

\begin{document}
\maketitle

\tableofcontents

\renewcommand{\abstractname}{Introduction} %changes the abstract to be called an introduction.
\begin{abstract}
This guide was written to explain how to use git (and GitHub) quickly and easily. While it definitely is not a particularly good guide, I couldn't find one online that explained, practically, how to use git with the right context and simply enough. So I wrote this. For an explanation of how Git works (which you \emph{must} know before you read this (knowing ``git commit'', ``git pull'' and ``git push'' or their equivalent GUI buttons dosen't count)) see \href{https://blog.red-badger.com/2016/11/29/gitgithub-in-plain-english}{Git \& Github in plain english} (seriously - do read it).
\end{abstract}

\section{Solving merge conflicts}
\label{sec:solving_merge_conflicts}

\section{Branching \& Pull Request Workflow}
\label{sec:branching_pull_request_workflow}
This explains a simple workflow, but in extreme detail. A quick outline of this workflow:
\begin{enumerate}
	\item Create a branch and work on your new changes there
	\item Create a pull request to incorporate changes into master
	\item Review \& approve the pull request, making changes as necessary
	\item Delete the branch
\end{enumerate}

\subsection{Creating a branch}
\label{sec:BPRW_creating_branch}
\begin{enumerate}
	\item Make sure your git repo is up to date with origin (ie Github) - either by running ``git pull'' or in the GUI clicking the ``Fetch origin'' button
	\item Create the branch by running ``git checkout -b your-branch-name'' or in the GUI going ``Branch \textgreater New Branch'' (also available from the ``current branch'' button next to fetch origin).
	\item Choose a suitable branch name, compliant with relevant conventions/guidelines - it should be short and descriptive. You should create a new branch for every self-contained feature - do NOT create large monolithic branches for miscellaneous updates; even if it means having a branch with just one commit.
	\item Add this branch to origin either by running ``git push --set-upstream origin your-branch-name'' (the extra option connects your new branch to a new reflected branch in origin) or hitting the ``Publish branch'' button in the GUI.
	\begin{figure}[h]
		\centering
		\includegraphics[width=\textwidth]{1publish-branch}
		\caption{The publish branch button in the GUI}
	\end{figure}
	\item Now you can start working on your feature, updating code, making commits and pushing them up to Github as normal!
	\begin{figure}[h]
		\centering
		\includegraphics[width=0.75\textwidth]{2modified-file}
		\caption{These important files have been modified (\&commited) in the branch}
	\end{figure}
\end{enumerate}

\subsection{Creating the pull request}
\label{BPRW_creating_pull_request}
\begin{enumerate}
	\item First, we must encorporate any changes that have been made to master while we have been working on our branch.
	\begin{enumerate}
		\item Update the local copy of master using ``git checkout master'' and ``git pull'' or in the GUI clicking the ``Fetch origin'' button
		\item Next, merge these changes into your branch by checking out your feature branch and running ``git merge master'' or in the GUI going ``Branch \textgreater Update from master''. You may need to resolve merge conflicts - if so, see the \hyperref[sec:solving_merge_conflicts]{Solving merge conflicts} section.
	\end{enumerate}
	\item sync your merge changes with origin (``git push'' or the sync button)
	\item Go to the github website for the repository, changing to your feature branch
	\begin{figure}[h]
		\centering
		\includegraphics[width=0.8\textwidth]{3github-website}
		\caption{Note that we have selected the repository, and the feature branch}
	\end{figure}
	\item Create the pull requst
	\begin{enumerate}
		\item Click on the ``pull request'' button
		\begin{figure}
			\centering
			\includegraphics[width=\textwidth]{4start-pull-request}
			\caption{Click the ``pull request'' button to create the pull request}
		\end{figure}
		\item Check the branches are correct (master on left, feature branch on right)
		\item create a helpful title and comment
		\item Review the code changes shown below
		\item Create the pull request
		\begin{figure}
			\centering
			\includegraphics[width=\textwidth]{5create-pull-request}
			\caption{Do the things}
		\end{figure}
	\end{enumerate}
	\item Now, just get a programming leader to review your pull request!
\end{enumerate}

\subsection{Reviewing the pull request}
\label{BPRW_reviewing_pull_request}

	


\subsubsection{Making changes}
\label{BPRW_RPW_making_changes}
temp

\subsubsection{Approving the pull request}
\label{BPRW_RPW_approving_pull_request}
temp

\subsection{cleanup}
\label{BPRW_cleanup}
temp

\end{document}